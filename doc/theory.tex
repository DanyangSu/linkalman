\documentclass[12pt]{article}
\usepackage{enumitem}
\usepackage{amssymb}
\usepackage{amsmath}
\usepackage{bbm}
\usepackage[a4paper, total={6in, 8in}]{geometry}
\usepackage[style=authoryear,sorting=nyt]{biblatex}
\newtheorem{lemma}{Lemma}
\addbibresource{ref.bib}
\setlength{\parindent}{0cm}
\setlength{\parskip}{1em}

\newenvironment{boenumerate}
    {\begin{enumerate}\renewcommand\labelenumi{\textbf\theenumi}}
    {\end{enumerate}}
\numberwithin{equation}{section}
\begin{document}
\title{Notes for Kalman Filter EM algorithms}

\section{Introduction:}
\texttt{linkalman} is a python package that solves linear structural time series models with Gaussian noises. Compared with some other popular Kalman filter packages written in python, \texttt{linkalman} has a combination of several advantages:
\begin{boenumerate}
    \item Account for incomplete measurements 
    \item Flexible and convenient model structure
    \item Robust and efficient implementation
    \item Exact initialization
    \item Built-in EM algorithm solver
    \item Open-source
    \item Pure python implementation 
\end{boenumerate}
Kalman filtering is a technique that provides an elegant solution to a wide range of time series problems. When I started learning Kalman filtering, I found most many existing Kalman filter packages are based on standard textbook models that are over-simplified for pedagogical purpose. In practice, a dynamic system may assume complex functional forms and may have incomplete measurements. In addition, solving a Kalman filter requires knowledge of initial conditions, which is rarely satisfied in real world problems. Finally, numerical implementation of Kalman filter algorithms are vulnerable to failures from rounding errors. \texttt{linkalman} package provides a solid solution to all these challenges. 

This document is a product of my learning on Kalman filter\footnote{The primary reference is the wonderful textbook by (\cite{durbin_koopman_2001}). I also refer to many other papers for technical details omitted in the textbook.} and serves as the theoretical foundation and user manual of \texttt{linkalman}. Section (\ref{sec:model_setup}) lays out the general structure of a Hidden Markov Model (HMM). Section (\ref{sec:filter}) and (\ref{sec:smoother}) provide detailed discussions on Kalman filters and Kalman smoothers, respectively. Section (\ref{sec:EM}) presents the EM algorithm to estimate underlying parameters of a HMM, Section (\ref{sec:codebase}) explains the package design of \texttt{linkalman}, and Section (\ref{sec:apply}) presents applications of Kalman filtering. 

\section{Model Setup:} \label{sec:model_setup}
Consider the following Linear Dynamic System:
\begin{align}
    \xi_{t+1} = & F_{t}\xi_{t} + B_{t}x_t + v_t \label{eq:state_evolve} \\
    y_t = & H_t\xi_{t} + D_{t}x_t + w_t \label{eq:measure}
\end{align}
Equation (\ref{eq:state_evolve}) governs a Markov state transition process. $\xi_t$ $(m\times 1)$ is the latent state random vector at time $t$, $x_t$ $(k\times 1)$ is the deterministic input signal, $v_t$ $(m\times 1)$ is the exogenous process noise. We assume that $v_t\sim \mathcal{N}(0,Q_t)$ is white noise\footnote{For process noises that are not white noise, we can re-write equation (\ref{eq:state_evolve}) to maintain independence across time. See Section (\ref{sec:apply}) for examples}. $F_t$ $(m\times m)$, $B_t$ $(m\times k)$, and $v_{t+1}$ specify the transition dynamics between $t$ and $t+1$. 

Equation (\ref{eq:measure}) is the measurement specification. $y_t$ $(n\times 1)$ is the measurement vector at time $t$. $w_t$ $(n\times 1)$ is the exogenous measurement noise. In addition to assuming $w_t\sim \mathcal{N}(0, R_t)$ is white noise, I also assume that  $w_t \perp v_s \ \forall\  t,s\in\{0,1,...,T\}$. $H_t$ $(n\times m)$ and $D_t$ $(n\times k)$ dictate interaction among $\xi_t$, $y_t$ and $x_t$. 

Equations (\ref{eq:state_evolve}) and (\ref{eq:measure}) characterize an HMM, with system matrices $M_t$ defined as:
\begin{align*}
M_t\equiv\{F_t, B_t, H_t, D_t, Q_t, R_t\}
\end{align*}
The subscript $t$ allows flexible model specification. For example, regression effects in time series models are placed in $B_t x_t$; an ARMA process can be modeled by $F_t\xi_t$ and $v_t$; additive outliers fit into $B_t x_t$. We bind $M_t$ with some underlying parameter set $\theta$. For example cyclical pattern can be modeled by using Trigonometric Cycles (\cite{harvey_1985}).  

\section{Kalman Filter} \label{sec:filter}
\subsection{Filtering with Known Initial Conditions:}
Given a set of measurements over time, and suppose $M_t$ is known. To predict $y_{T+1}$, we need to perform forward filtering. We start by making the following notations:
\begin{align*}
    Y_t &\equiv \{y_1, y_2, ..., y_t\} \\
    X_t &\equiv \{x_1, x_2, ..., x_t\} \\
    \Xi_t &\equiv \{\xi_1,\xi_2,...,\xi_t\} \\
    \hat{\xi}_{t,s} &\equiv E(\xi_t|X_{s},Y_{s};\theta) \\
    P_{t,s} &\equiv E[(\xi_t-\hat{\xi}_{t,s})(\xi_t-\hat{\xi}_{t,s})']
\end{align*}
With the assumption that $w_t$ and $v_t$ are Gaussian white noises, conditional distribution of $y_t$ and $\xi_t$ are fully characterized by Gaussian process with $\hat{\xi}_{t,s}$ and $P_{t,s}$. Now consider at time $t$, we know $Y_t$ and $M_t$, and suppose we also know $\hat{\xi}_{t,t-1}$ and $P_{t,t-1}$. By equation (\ref{eq:state_evolve}), we have\footnote{The derivation closely follows Chapter 13 in (\cite{hamilton_1994}). In particular, derivation of equation (\ref{eq:p_tt}) is given in Appendix \ref{ap:iter_proj}.}:
\begin{align}
    \hat{\xi}_{t,t} &= \hat{\xi}_{t,t-1} + K_t(y_t-H_t\hat{\xi}_{t,t-1}-D_tx_t) \label{eq:filter_begin} \\
    K_t &= P_{t,t-1}H_t^{'}\Upsilon_t^{-1} \label{eq:gain} \\
    \Upsilon_t &\equiv H_tP_{t,t-1}H_t^{'} + R_t \\
    P_{t,t} &= P_{t,t-1} - K_tH_tP_{t,t-1} \label{eq:p_tt} \\
    \hat{\xi}_{t+1,t} &= F_t\hat{\xi}_{t,t} + B_tx_t \label{eq:xi_t1} \\
    P_{t+1,t} &= F_tP_{t,t}F_t^{'}+Q_t \label{eq:filter_end}
\end{align}
$K_t$ is the famous Kalman gain matrix. In this subsection, we assume that initial conditions $\hat{\xi}_{1,0}$ and $P_{1,0}$ are known. The Kalman filter proceeds as follows:
\begin{boenumerate}
    \item Given $\theta$, begin with initial value $\hat{\xi}_{1,0}$ and $P_{1,0}$.
    \item Use equation (\ref{eq:filter_begin}) through (\ref{eq:filter_end}) to calculate $\hat{\xi}_{2,1}$, and $P_{2,1}$.
    \item Repeat step 2 for $t\in\{3, 4, ..., T\}$.
\end{boenumerate}

\subsection{Joseph Form and Numerical Robustness}
We define $P_{t,t}$ as a covariance matrix, which is a symmetric positive semi-definite (PSD) matrix, but in practice, we may get $P_{t,t}$ that is neither symmetric nor PSD, due to rounding errors. Look at equation (\ref{eq:p_tt}) again. The subtraction makes calculating $P_{t,t}$ susceptible to rounding errors, sometimes even resulting in negative semi-definite matrix. I use the Joseph form to numerically enforce symmetric PSD. 

The Joseph form (\cite{joseph_1968}) of $P_{t,t}$ is\footnote{For a detailed proof, see Appendix \ref{ap:joseph}}:
\[
    P_{t,t} = [I - K_tH_t]P_{t,t-1}[I - K_tH_t]' + K_tR_tK_t^{'}    
\]
If we guarantee $P_{t,t-1}$ and $R_t$ to be PSD, then $P_{t,t}$ is PSD by construction. Define $M_t$, $L_t$, and $d_t$ as:
\begin{align*}
    M_t &\equiv F_tK_t \\
    L_t &\equiv F_t-M_tH_t \\
    d_t &\equiv y_t - H_t\hat{\xi}_{t,t-1} - D_tx_t
\end{align*}
We are able to obtain a more concise updating rule:
\begin{align}
    \hat{\xi}_{t+1,t} &= F_t\hat{\xi}_{t,t-1} + M_td_t \label{eq:joseph_update_xi} \\
    P_{t+1,t} &= L_tP_{t,t-1}L_t^{'} + M_tR_tM_t^{'} + Q_t \label{eq:joseph_update_P}
\end{align}
Using equation (\ref{eq:joseph_update_xi}) through (\ref{eq:joseph_update_P}) recursively, we may implement Kalman filtering in the Joseph form. 

\subsection{Initialization with Diffuse Filtering} \label{subsec:diff_filter}
So far we have assumed that we know values of $\hat{\xi}_{1,0}$ and $P_{1,0}$. In practice, such information are rarely available. If instead, we know that the time series is stationary, we can use the unconditional mean and variance for $\hat{\xi}_{1,0}$ and $P_{1,0}$. If the time series is at least partially non-stationary, then we should assume no prior information and set $\hat{\xi}_{1,0}=0$ and $P_{1,0}=\infty$ for the non-stationary part of $\xi_t$. A commonly used solution is to set $P_{1,0}$ to some large value (e.g. $\kappa=10^7$) to approximate infinity, but this technique is difficult to implement when the initial condition structure is complex\footnote{See Section 3 of (\cite{doan_2010}) for further discussions.}. Better algorithms exist for exact initialization, and I will discuss one such algorithm that is implemented by \texttt{linkalman}.

Consider $\hat{\xi}_{1,0}$ again. Following (\cite{koopman_1997}), we define $\hat{\xi}_{1,0}$ as:
\begin{align}
    \hat{\xi}_{1,0} = a + A\eta + \Pi\varepsilon \label{eq:init}
\end{align}
$A$ and $\Pi$ are selection matrices, $\eta\sim\mathcal{N}(0,\kappa I_{\infty})$, and $\varepsilon\sim\mathcal{N}(0,Q_{*})$. Set $0$ for elements in $a$ corresponding to $A$, and stationary means for elements corresponding to $\Pi$. $I_{\infty}$ is an identity matrix with size equal to the number of non-stationary unknowns, and $Q_{*}$ is Essentially, equation (\ref{eq:init}) groups $\hat{\xi}_{1,0}$ into two categories. If $\hat{\xi}_{1,0}^i$, the $i$-th element of $\hat{\xi}_{1,0}$, is non-stationary and not known, it has a distribution $\mathcal{N}(0, \kappa)$. With $\kappa\rightarrow\infty$, $\mathcal{N}(0,\kappa)$ captures the fact that we know nothing about the initial value of a non-stationary process. If on the other hand, the initial state of $\hat{\xi}_{1,0}^i$ is either known or stationary, It has a proper distribution with unconditional mean and covariances\footnote{If $X_t$ is involved in the state transition process then most of the time the process is not stationary.}. 

From equation (\ref{eq:init}), we have:
\begin{align}
    \hat{\xi}_{1,0} &= a  \label{eq:init_xi}\\
    P_{1,0} &= \kappa P_{\infty} + P_{*} \label{eq:init_P}
\end{align}
where $P_{\infty}=AA^{'}$ and $P_{*}=\Pi Q_0 \Pi^{'}$. By linearity of the filtering process, we can write $P_{t,t-1}$ as:
\begin{align}
    P_{t,t-1} = \kappa P_{\infty,t} + P_{*,t} + \mathcal{O}(\kappa^{-1}) \label{eq:P_diffuse}
\end{align}
where $\mathcal{O}(\kappa^{-1})$ is a function $f(\kappa)<\infty$ as $\kappa\rightarrow\infty$. We can write $\Upsilon_{t}$ as:
\begin{align*}
    \Upsilon_t = \kappa\Upsilon_{\infty,t} + \Upsilon_{*,t} + \mathcal{O}(\kappa^{-1})
\end{align*}
In what follows, I describe the updating rule for exact initialization, for technical details, please refer to Appendix \ref{ap:init_filter}.

Given $\hat{\xi}_{t,t-1}$ and $P_{t,t-1}$, if $\Upsilon_{\infty,t}\neq 0$:
\begin{boenumerate}
    \item Calculate $K_t$ using equation (\ref{eq:K1_diffuse_start}) to (\ref{eq:K1_diffuse_end})
    \item Calculate $\hat{\xi}_{t,t}$ and $P_{t,t}$ from equation (\ref{eq:diff_xi1}) and (\ref{eq:diff_P1})
\end{boenumerate}

If $\Upsilon_{\infty,t}=0$:
\begin{boenumerate}
    \item Calculate $K_t$ using equation (\ref{eq:K2_diffuse})
    \item Calculate $\hat{\xi}_{t,t}$ and $P_{t,t}$ from equation (\ref{eq:diff_xi2}) and (\ref{eq:diff_P2})
\end{boenumerate}

We can update the diffuse Kalman filter with the new expressions for $K_t$, $\hat{\xi}_{t+1,t}$, and $P_{t+1,t}$.

\subsection{Transition to the Usual Kalman Filter}
Diffused initial state is to represent the fact the we have no prior knowledge of initial the state of non-stationary processes. As a result, that state estimate will be dominated by initial data. In other words, the few initial observations are used for form the informative initial conditions, and the rest of the data are used to perform usual Kalman filtering. (\cite{dejong_1991}) and (\cite{durbin_koopman_2003}) show that a diffuse Kalman filter will degenerate to a usual Kalman filter after a few time periods $t_q$, which in general is the number of diffuse initial states. 

A diffuse Kalman filter degenerates to a usual one if $P_{\infty,t}=0$. In practice, we may check if $P_{\infty,t}=0$, but it is subject to numerical errors. (\cite{helske_2016}) implemented a more robust algorithms in his R package \texttt{KFAS}. I provide the procedure below, and interested readers may refer to Appendix \ref{ap:transition} for a simple proof\footnote{For a complete treatment, refer to \cite{koopman_1997} for details.}. 

Here I focus on the univariate measurement case, because it is less complex and convertible from multivariate Kalman filters. Define $q_t\equiv rank(P_{\infty,t})$, then by construction of $P_{1,0}$, $q_1$ has rank $q$, where $q$ is the number of diffuse state variables. Let $\varepsilon_t>0$ be some small threshold value\footnote{In practice, I follow (\cite{helske_2016}) and set $\varepsilon_t=\varepsilon\times min(|H_t|;|H_t|>0)^2$,  where $\varepsilon$ is some base threshold, and $min(|H_t|;|H_t|>0)$ is the minimum of the absolute values in $H_t$ that is not $0$.} such that if $\Upsilon_{\infty,t}<\varepsilon_t$, we treat $\Upsilon_{\infty,t}=0$. If $\Upsilon_{\infty,t}\geq \varepsilon_t$, we decrease $q$ by $1$. After each period, we set: 
\[
    q_{t+1}=min[rank(P_{\infty,t,t}),rank(F_t), rank(F_tP_{\infty,t,t}F_t^{'})]
\]
When $q_{t}=0$ for $t=t_q$, the diffuse Kalman filter degenerates into a usual one. 

\subsection{Missing Measurements and Sequential Filtering} \label{subsec:seq_filter}
So far we have assumed that we observe complete measurement $y_t$ for each $t$. If we have incomplete measurements, we can instead update the Kalman Filter sequentially (see (\cite{durbin_koopman_2001}) section 6.4 for details), based only on observed measurements. Sequential filtering also boosts speed dramatically in the presence of large measurement dimensions, reducing cost from $\mathcal{O}(n^3)$ to $\mathcal{O}(n)$. The techniques developed in this subsection is implemented in \texttt{linkalman} and will be discussed in detail. 

Denote $y_{t:(i)}$ as the $i$-th observed measurement at time $t$, then $y_{t:(1)}$ is the first non-missing univariate measurement at time $t$. Let $(n_t)$ as the index for the last non-missing univatriate measurement at time $t$. Define $\hat{\xi}_{t:(i)}$ and $P_{t:(i)}$ as the state estimates after updating univariate measurement $(i)$ at time $t$:
\begin{align*}
    \hat{\xi}_{t:(i)} &\equiv E(\xi_t|Y_{t-1},y_{t:(1)},...,y_{t:(i-1)};\theta) \\
    P_{t:(i)} &\equiv Var(\xi_t|Y_{t-1},y_{t:(1)},...,y_{t:(i-1)};\theta) 
\end{align*}

In addition, we use $(0)$ as index of the state estimates before any updating, then we have:
\begin{align}
    \hat{\xi}_{t:(0)} &\equiv E(\xi_{t}|Y_{t-1};\theta)\nonumber \\
    &= F_{t-1}\hat{\xi}_{t-1:(n_{t-1})}+B_{t-1}x_{t-1} \label{eq:diff_xi_seq0} \\
    P_{t:(0)} &\equiv Var(\xi_{t}|Y_{t-1};\theta) \nonumber \\
    &= F_{t-1}P_{t-1:(n_{t-1})}F_{t-1} + Q_{t-1} \label{eq:diff_P_seq0}
\end{align}

For sequential filtering to work, we need $R_t$ to be diagonal. If $R_t$ is not diagonal, we can use LDL Decomposition\footnote{In practice, I perform LDL Decomposition with \texttt{scipy.linalg.ldl}.} to transform the original HMM into one with independent measurement noise. Given that $R_t$ is PSD, we have $R_t = l_t\Lambda_tl_t^{'}$. Pre-multiply equation (\ref{eq:measure}) by $l_t^{-1}$, and we have\footnote{Due to special properties of triangular matrices, I perform matrix inverse on triangular matrix with \texttt{scipy.linalg.lapack.clapack.dtrtri} subroutine.}:
\begin{align}
    \tilde{y}_t = \tilde{H}_t\xi_{t} + \tilde{D}_{t}x_t + \tilde{w}_t \label{eq:ldl}
\end{align}
where $\tilde{(\cdot)}_t = l_t^{-1}(\cdot)_t$. In the following sections, I will omit the $(\sim)$ sign, and assume $R_t$ is always diagonal.

For an HMM with diagonal $R_t$, equation (\ref{eq:measure}) then becomes:
\begin{align*}
    y_t &= 
    \begin{pmatrix}
        y_{t:(1)} \\
        \vdots \\ 
        y_{t:(n_t)}
    \end{pmatrix} 
    = \begin{pmatrix}
        H_{t:(1)}\xi_t + D_{t:(1)}x_t + w_{t:(1)} \\
        \vdots \\
        H_{t:(n_t)}\xi_t + D_{t:n_t}x_t + w_{t:(n_t)}
    \end{pmatrix}
\end{align*}
where $H_{t:(i)}$ and $D_{t:(i)}$ are the $(i)$-th row of $H_t$ and $D_t$. 

We initialize the measurement update process with $\hat{\xi}_{1,0}$ and $P_{1,0}$ from equation (\ref{eq:init_xi}) and (\ref{eq:init_P}). Define $\Upsilon_{t:(i)}$ as:
\begin{align*}
    \Upsilon_{t:(i)} = H_{t:(i)}P_{t:(i-1)}H_{t:(i)}^{'} + R_{t:(i)}
\end{align*}
where $R_{t:(i)}$ is the $(i)$-th diagonal value of $R_{t}$. For $\Upsilon_{t:(i)}$, we have:
\begin{align*}
    \Upsilon_{\infty, t:(i)} &= H_{t:(i)}P_{\infty, t:(i-1)}H_{t:(i)}^{'} \\
    \Upsilon_{*,t:(i)} &= H_{t:(i)}P_{*,t:(i-1)}H_{t:(i)}^{'} + R_{t:(i)}
\end{align*}

For each successive measurement $i$, if $\Upsilon_{\infty,t:(i)}\neq0$, we have:
\begin{align}
    \hat{\xi}_{t:(i)} = & \hat{\xi}_{t:(i-1)} + K_{t:(i)}^{(0)}d_{t:(i)} + \mathcal{O}(\kappa^{-1}) \label{eq:diff_seq_xi1} \\
    P_{t:(i)} =& \kappa(I-K_{t:(i)}^{(0)}H_{t:(i)})P_{\infty,t:(i-1)}(I-K_{t:(i)}^{(0)}H_{t:(i)})^{'} \label{eq:diff_seq_P1} \\
        & + (I-K_{t:(i)}^{(0)}H_{t:(i)})P_{*,t:(i-1)}(I-K_{t:(i)}^{(0)}H_{t:(i)})^{'} \nonumber \\
        & + K_{t:(i)}^{(0)}R_{t:(i)}K_{t:(i)}^{(0)'} + \mathcal{O}(\kappa^{-1}) \nonumber
\end{align}
where
\begin{align*}
    d_{t:(i)} &= y_{t:(i)} - H_{t:(i)}\hat{\xi}_{t:(i-1)}-D_{t:(i)}x_t \\
    K_{t:(i)}^{(0)} &= \frac{P_{\infty,t:(i-1)}H_{t:(i)}^{'}}{\Upsilon_{\infty,t:(i)}} 
        =\frac{P_{\infty,t:(i-1)}H_{t:(i-1)}^{'}}{H_{t:(i)}P_{\infty, t:(i-1)}H_{t:(i)}^{'}}
\end{align*}

If $\Upsilon_{\infty,t:(i)}=0$ and $P_{\infty,t:(i-1)}\neq 0$, we have:
\begin{align}
    \hat{\xi}_{t:(i)} =& \hat{\xi}_{t:(i-1)} + K_{t:(i)}^{(*)}d_{t:(i)} + \mathcal{O}(\kappa^{-1}) \label{eq:diff_seq_xi0} \\
    P_{t:(i)} =& \kappa P_{\infty,t:(i-1)} + (I-K_{t:(i)}^{(*)}H_{t:(i)})P_{*,t:(i-1)}(I-K_{t:(i)}^{(*)}H_{t:(i)})^{'} \label{eq:diff_seq_P0} \\
        &+ K_{t:(i)}^{(*)}R_{t:(i)}K_{t:(i)}^{(*)'} + \mathcal{O}(\kappa^{-1}) \nonumber 
\end{align}
where
\begin{align*}
    K_{t:(i)}^{(*)} &= \frac{P_{*,t:(i-1)}H_{t:(i)}^{'}}{\Upsilon_{*,t:(i)}} = \frac{P_{*,t:(i-1)}H_{t:(i)}^{'}}{H_{t:(i)}P_{*,t:(i-1)}H_{t:(i)}^{'} + R_{t:(i)}}
\end{align*}

If $\Upsilon_{\infty,t:(i)}=0$ and $P_{\infty,t:(i-1)}=0$, the diffuse Kalman filter degenerates into a usual one, and instead of using equation (\ref{eq:diff_seq_P0}), we use:
\begin{align}
    P_{t:(i)} =& (I-K_{t:(i)}^{(*)}H_{t:(i)})P_{*,t:(i-1)}(I-K_{t:(i)}^{(*)}H_{t:(i)})^{'} + K_{t:(i)}^{(*)}R_{t:(i)}K_{t:(i)}^{(*)'} \label{eq:usual_seq_P0} 
\end{align}

Now we may proceed with sequential filtering as follows:
\begin{boenumerate}
    \item Initialize state conditions using equation (\ref{eq:init_xi}) and (\ref{eq:init_P}) 
    \item For period $t$, calculate $\hat{\xi}_{t:(0)}$ and $P_{t:{(0)}}$ using equation (\ref{eq:diff_xi_seq0}) and (\ref{eq:diff_P_seq0}) \label{step:seq_filter_begin}
    \item Use LDL transformation to obtain diagonalized measurement equation (\ref{eq:ldl})
    \item If $\Upsilon_{\infty,t:(i)}\neq0$, use equation (\ref{eq:diff_seq_xi1}) and (\ref{eq:diff_seq_P1}) to update $\hat{\xi}_{t:(i)}$ and $P_{t:(i)}$
    \item If $\Upsilon_{\infty,t:(i)}=0$ and $P_{\infty,0}\neq0$, use equation (\ref{eq:diff_seq_xi0}) and (\ref{eq:diff_seq_P0}) to update $\hat{\xi}_{t:(i)}$ and $P_{t:(i)}$ 
    \item If $\Upsilon_{\infty,t:(i)}=0$ and $P_{\infty,0}=0$, use equation (\ref{eq:diff_seq_xi0}) and (\ref{eq:usual_seq_P0}) to update $\hat{\xi}_{t:(i)}$ and $P_{t:(i)}$ 
    \item If $t<t_q$, update $q_{t+1} = min[rank(P_{\infty,t,t}), rank(F_t)]$. \label{step:seq_filter_end}
    \item Repeat step (\ref{step:seq_filter_begin}) through (\ref{step:seq_filter_end}) for $t\in\{1,2,...,T\}$
\end{boenumerate}

\section{Kalman Smoother} \label{sec:smoother}
\subsection{State Smoother}
In Section \ref{sec:filter}, we use Kalman Filter to find $\{\hat{\xi}_{t,t}, K_t, P_{t,t}, \hat{\xi}_{t,t-1}, P_{t,t-1}\}$ for each $t$. If a dataset is given, we have the entire measurement sequence $Y_T$. Kalman Smoother is a technique of integrating information up to $T$ to infer $\xi_t$ at time $t$, $\hat{\xi}_{t,T}$ and $P_{t,T}$. 

Suppose in addition to state estimates from Kalman Filter, we also want to know $\hat{\xi}_{t+1,T}$ and $P_{t,T}$. The technique for computing $\hat{\xi}_{t+1,T}$ and $P_{t,T}$ is called backwards smoothing\footnote{See (\cite{dejong_1989}) for details. I also provided a proof in Appendix \ref{ap:smooth} that is consistent with notations in this document and has more details.}. Here I present the iterative formula for Kalman smoothing. Following derivations in Appendix \ref{ap:smooth}, I define two auxiliary variables: $r_{t}$ and $N_{t}$, where $r_T=0$, $N_T=0$, and:
\begin{align}
    r_{t-1} &= H_t^{'}\Upsilon_t^{-1}d_t + L_t^{'}r_t \label{eq:r1t} \\
    N_{t-1} &= H_t^{'}\Upsilon_t^{-1}H_t + L_t^{'}N_tL_t \label{eq:N1t}
\end{align}

Using $\{r_t\}_{1,...,T}$ and $\{N_t\}_{1,...,T}$, we have the iterative formulation for $\hat{\xi}_{t,T}$ and $P_{t,T}$:
\begin{align}
    \hat{\xi}_{t,T} &= \hat{\xi}_{t,t-1} + P_{t,t-1}r_{t-1} \label{eq:smooth_state2} \\
    P_{t,T} &= P_{t,t-1}- P_{t,t-1}N_{t-1}P_{t,t-1} \label{eq:smooth_P2}
\end{align}

To perform backwards smoothing, we can simply start at time $T$ with $r_T=0$ and $N_T=0$, then use equation (\ref{eq:r1t}) through (\ref{eq:smooth_P2}) for $t\in\{T-1,T-2,...,1\}$ to obtain $\{\hat{\xi}_{t,T}\}_{1,2,...,T}$ and $\{P_{t,T}\}_{1,2,...,T}$.

\subsection{Disturbance Smoother}
In this subsection, I will discuss disturbance smoother, a variation of the Kalman smoother that provides more efficient smoothed state estimates and clean forms for calculating likelihoods\footnote{This methods does not provide estimates of $P_{t,T}$. If finding $P_{t,T}$ is the goal, use equation (\ref{eq:r1t}) through (\ref{eq:smooth_P2}) instead}. 

Consider the smoothed state disturbance estimates:
\begin{align*}
    \hat{v}_{t,T} &\equiv E(v_t|Y_T,X_T;\theta) \\
    \hat{w}_{t,T} &\equiv E(w_t|Y_T,X_T;\theta) \\
    V_t &\equiv Var(v_t|Y_T,X_T;\theta) \\
    W_t &\equiv Var(w_t|Y_T,X_T;\theta)
\end{align*}

Following derivations in Appendix \ref{ap:disturb_smooth}, we have the iterative formulation:
\begin{align}
    \hat{w}_{t,T} &= R_t(\Upsilon_t^{-1}d_t - M_t^{'}r_t) \label{eq:disturb_w} \\
    \hat{v}_{t,T} &= Q_tr_t \label{eq:disturb_v} \\
    W_t &= R_t - R_t(\Upsilon_t^{-1}+M_t^{'}N_tM_t)R_t \label{eq:disturb_W} \\
    V_t &= Q_t - Q_tN_tQ_t \label{eq:disturb_V}
\end{align}

Using equation (\ref{eq:disturb_w}) through (\ref{eq:disturb_V}), we can get estimates for smoothed disturbances through backwards recursion.

\subsection{Diffuse Disturbance Smoothing}
In this subsection, I discuss smoothing when initial conditions are not fully known. From section \ref{subsec:diff_filter}, we know that $P_{\infty,t}=0 \ \forall\ t\geq t_q$ after period $t_q$ where a diffuse Kalman filter degenerates into a usual one, and we can proceed with standard Kalman filtering. Therefore, for $t\in \{T,T-1,...,t_q\}$, we can proceed with standard Kalman smoothing as well. For $t\in\{t_q-1,t_q-2,...,1\}$, we need to perform diffuse smoothing.

For $t\in\{T,T-1,...,t_q\}$, we can perform usual disturbance smoothing where $r_t=r_t^{(0)}$ and $N_t=N_t^{(0)}$. For $t\in\{t_q-1,t_q-2,...,1\}$, if $\Upsilon_{\infty,t}\neq0$, we use equation (\ref{eq:diff_disturb1_start}) through (\ref{eq:diff_disturb1_end}) for diffuse disturbance smoothing. If $\Upsilon_{\infty,t}=0$, we use equation (\ref{eq:diff_disturb0_start}) through (\ref{eq:diff_disturb0_end}) for diffuse disturbance smoothing. 

\subsection{Rounding Errors and Nearest PSD}
Note that for Kalman Smoother, we don't have Joseph Form formulation, and therefore we may still get non-PSD covariance matrix and cause the smoother to fail. By construction, I guarantee symmetry of covariance matrices. Therefore, we may check PSD using Cholesky Decomposition, which is efficient under symmetry conditions.

If a covariance matrix $A$ is not PSD, we can use the nearest PSD matrix with the techniques developed by (\cite{higham_1988}). Omitting technical details, it amounts to replacing $A$ with $\bar{A}$, where $\bar{A}$ is be calculated as:
\begin{align*}
    A &= USV^{'} \\
    \Sigma &= VSV^{'} \\
    \bar{A} &= \frac{\Sigma + \Sigma^{'} + A + A^{'}}{4}
\end{align*}
where the first equality is the SVD decomposition of $A$.

\subsection{Sequential Smoother}
As with the case of Kalman filtering, calculating $\Upsilon_t^{-1}$ is expensive and subject to failure due to rounding errors. \cite{durbin_koopman_2000} propose sequential smoothing to greatly improved the efficiency and robustness of Kalman Smoothers. The univariate approach in Kalman filter essentially treats observations flowing in one at a time, so we can readily adapt our smoother.

Define $y_{t,i}$ as the $i$-th non-missing measurement\footnote{Note that $i$ may not correspond to the index of the measurement if some are missing.} of $y_t$, we have:
\begin{align}
    r_{t:i-1} &= \begin{cases}
        \frac{H_{t:i}^{'}d_{t:i}}{\Upsilon_{t:i}} + L_{t:i}^{'}r_{t:i} \\
        F_{t-1}^{'}r_{t:0} 
    \end{cases} \label{eq:seq_r} \\
    \Upsilon_{t:i} &= H_{t:i}P_{t,t:i-1}H_{t:i}^{'}+\sigma_{t:i}^2 \nonumber \\
    L_{t:i} &= F_t(I-K_{t:i}H_{t:i}) \nonumber \\
    N_{t:i-1} &= \begin{cases}
        \frac{H_{t:i}^{'}H_{t:i}}{\Upsilon_{t:i}} + L_{t:i}^{'}N_{t:i}L_{t:i} \\
        F_{t-1}^{'}N_{t:0}F_{t-1}
    \end{cases} \label{eq:N_seq}
\end{align}

For disturbance smoother, we may calculate $\hat{v}_{t,T:i}$, $V_{t:i}$, $\hat{w}_{t,T:i}$, and $W_{t:i}$ as:
\begin{align}
    \hat{w}_{t,T:i} &= R_{t:i}(\frac{d_{t:i}}{\Upsilon_{t:i}}-K_{t:i}^{'}H_{t:i}^{'}r_{t:i}) \\
    W_{t:i} &= R_{t:i} - R_{t:i}^2(\Upsilon_{t:i}^{-1}+K_{t:i}^{'}F_{t:i}^{'}N_{t:i}F_{t:i}^{'}K_{t:i})  \\
    \hat{v}_{t,T:i} &= Q_tr_{t:i} \\
    V_{t:i} &= Q_t - Q_tN_{t:i}Q_t
\end{align}

\section{EM Algorithm} \label{sec:EM}
Kalman Filter and Kalman Smoother are useful to predict measurements if we know $\theta$. Quite often, we need to estimate $\theta$ as well. A popular method that I implement here is EM algorithm\footnote{There are many other algorithms as well. For example, Newton-Raphson, and MCMC algorithm. EM algorithm is notable for its ease to implement. But it is difficult to generate a confidence interval of the parameters, unlike methods such as Newton-Raphson algorithm. For large enough time series data and decent parameter dimensions, this negligence does not affect the confidence of the prediction by much.}. For general proof, one can refer to Appendix \ref{ap:EM_proof}.

The log likelihood function for a HMM system is:
\begin{align*}
    L(Y_T,X_T, \theta) &= log[\mathbb{P}(Y_T|X_T,\theta) \\
\end{align*}
Following equation (\ref{eq:Q}), we maximize $L(Y_T,X_T,\theta)$ by maximizing: 
\begin{align*}
    G(Y_T,X_T,\theta) = \int log[\mathbb{P}(Y_T,\Xi_T|X_T,\theta)]\mathbb{P}(\Xi_T|Y_T,X_T,\theta)d\Xi_T 
\end{align*}
Denote $G(\theta,\theta_1)$ as: 
\[
    G(\theta,\theta_i) \equiv \int log[\mathbb{P}(Y_T,\Xi_T|X_T,\theta)]\mathbb{P}(\Xi_T|Y_T,X_T,\theta_i)d\Xi_T
\]
EM algorithm proceeds as follows:
\begin{boenumerate}
    \item Start with initial parameter value $\theta_0$
    \item \label{step:EM_E} For iteration $i$, use Kalman Smoother to get $\mathbb{P}(\Xi_T|Y_T,X_T,\theta_{i-1})$
    \item \label{step:EM_M} Find $\theta_{i}$ that maximizes $G(\theta,\theta_{i-1})$    
    \item Repeat step \ref{step:EM_E} and \ref{step:EM_M} until $\{G(Y_T,X_T,\theta_i)\}_i$ converges to a local optimal.
\end{boenumerate}

\subsection{$G(Y_T,X_T,\theta)$ with Missing Measurements} \label{subsec:G}
If we know $(Y_T,\Xi_T)$, by Markov Property, we can rewrite $log[\mathbb{P}(Y_T,\Xi_T|X_T,\theta)]$ as:
\begin{align}
    log[\mathbb{P}(Y_T,\Xi_T|X_T,\theta)] &= \sum_{t=1}^{T}log[\mathbb{P}(\xi_t|\xi_{t-1},x_{t-1},\theta)] 
    + \sum_{t=1}^{T}log[\mathbb{P}(y_t|\xi_t,x_t,\theta)] \nonumber \\
    &= log[\mathbb{P}(\xi_1|\theta)] + \sum_{t=2}^{T}log[\mathbb{P}(\xi_t|\xi_{t-1},x_{t-1},\theta)] \nonumber \\ 
    &+ \sum_{t=1}^{T}log[\mathbb{P}(y_t|\xi_t,x_t,\theta)]\label{eq:log}
\end{align}
For $t>1$, the second term in equation (\ref{eq:log}) is:
\begin{align}
    log[\mathbb{P}(\xi_t|\xi_{t-1},x_{t-1},\theta)] &= const-\frac{1}{2}log(|Q_t|) 
    -\frac{1}{2}\delta_t^{'}Q_t^{-1}\delta_t \nonumber \\
    &= const-\frac{1}{2}log(|Q_t|) 
    -\frac{1}{2}Tr[\delta_t^{'}Q_t^{-1}\delta_t] \nonumber \\
    &= const-\frac{1}{2}log(|Q_t|) 
    -\frac{1}{2}Tr[Q_t^{-1}\delta_t\delta_t^{'}] \label{eq:log1} \\
    \delta_t &\equiv \xi_t - F_{t-1}\xi_{t-1}-B_{t-1}x_{t-1} \nonumber
\end{align}
The second equality in equation (\ref{eq:log1}) holds because $Tr(a)=a$ for scalar $a$. The third equality holds because $Tr(AB)=Tr(BA)$. 

We may calculate:
\begin{align}
    G_1^{t}(\theta,\theta_{i-1}) &\equiv \int log[\mathbb{P}(\xi_t|\xi_{t-1},\theta)]\mathbb{P}(\Xi_T|Y_T,X_T,\theta_{i-1})d\Xi_T \nonumber \\
    &= const -\frac{1}{2}log(|Q_t|)-\frac{1}{2}Tr[E(Q_t^{-1}\delta_t\delta_t^{'}|Y_T,X_T,\theta_{i-1})] \nonumber \\
    &= const - \frac{1}{2}log(|Q_t|) - \frac{1}{2}Tr[Q_t^{-1}E(\delta_t\delta_t^{'}|Y_T,X_T,\theta_{i-1})] \label{eq:log1_trace}
\end{align}
Note that when $t=1$, we have $\xi_1\sim\mathcal{N}(\hat{\xi}_{1,0}, P_{1,0})$. We may calculate:
\begin{align}
    G_1^1(\theta,\theta_{i-1}) &= const - \frac{1}{2}log(|P_{1,0}|) \nonumber \\
    &- \frac{1}{2}Tr\{P_{1,0}^{-1}E[\delta_1\delta_1^{'}|Y_T, X_T, \theta_{i-1}]\} \nonumber \\
    \delta_1 &\equiv \xi_1 - \hat{\xi}_{1,0} \nonumber
\end{align}
Detailed derivation of $E(\delta_t\delta_t^{'}|Y_T,X_T,\theta_{i-1})$ is given in Appendix \ref{ap:log}. If $Q_t$ does not have full rank, I use \texttt{scipy.linalg.pinvh} to find the pseudo inverse of $Q_t$, or \texttt{scipy.linalg.eigh}\footnote{After finding all eigenvalues of $Q_t$, multiply non-zero eigenvalues to get the determinant.} to find the pseudo determinant of $Q_t$.

Calculating conditional expectation of the second term in equation (\ref{eq:log}) is similar. Let's consider:
\[
    G_2^t(\theta,\theta_{i-1}) \equiv \int log[\mathbb{P}(y_t|\xi_{t},x_t, \theta)]\mathbb{P}(\Xi_T|Y_T,X_T,\theta_{i-1})d\Xi_T 
\]
With the assumption of Gaussian white noises, we have:
\begin{align}
    G_2^t(\theta,\theta_{i-1}) &= const - \frac{1}{2}log(|R_t|)-\frac{1}{2}Tr[R_t^{-1}E(\chi_t\chi_t^{'}|Y_T,X_T,\theta_{i-1})] \label{eq:log2_trace} \\
    \chi_t &\equiv y_t - H_t\xi_t - D_tx_t \nonumber
\end{align}
See Appendix \ref{ap:log2} for detailed derivation of $G_2^t(\theta,\theta_{i-1})$. Now $G(\theta,\theta_{i-1})$ is:
\begin{align*}
    G(\theta,\theta_{i-1}) &= \sum_{t=1}^{T}G_1^t(\theta,\theta_{i-1}) + \sum_{t=1}^{T}G_2^t(\theta,\theta_{i-1})
\end{align*}
We can use numerical methods\footnote{If system matrices are constant, we have a closed form solution. But I decide to use numerical optimizations so that the EM is also able to solve HMM with complex system matrices.} to find $\theta_i$ that maximizes $G(\theta,\theta_{i-1})$.

\section{\texttt{linkalman}} \label{sec:codebase}

\section{Application} \label{sec:apply}
In this section, I use the Kalman Filter to solve a complex time series problem from (\cite{brodersen_etal_2015}) (with some modification). Consider the following time series: 
\begin{align}
    g_t &= \mu_t + \gamma_t + D_1x_t + w_{g,t} \label{eq:ts_setup} \\
    e_t &= \pi(\mu_t + \gamma_t) + D_2x_t + w_{e_t}
\end{align}
In equation (\ref{eq:ts_setup}), $g_t$ is the daily S\&P index, and $e_t$ is gold price index at time $t$, both of wich can be viewed as measurements of economic status at time $t$. We consider $g_t$ as the unbised measurement of $(\mu_t+\gamma_t)$, and $e_t$ is a biased measurement. $x_t$ includes $1$ as well as whether it is a holidy at time $t$. $\mu_t$ is the trend component, and $gamma_t$ is the seasonal component. $\mu_t$ has the following specification:
\begin{align}
    &\mu_{t+1} = \mu_{t} + \delta_t + \eta_{\mu,t+1} \\
    &\delta_{t+1} = \Delta_{\delta} + \rho_{\delta}(\delta_t - \Delta_{\delta}) + \eta_{\delta,t+1} 
\end{align}
where $\eta_{\mu,t}\sim\mathcal{N}(0,\sigma_{\mu}^2)$, $\eta_{\delta,t}\sim\mathcal{N}(0,\sigma_{\delta}^2)$. $\delta_t$ is interpreted as the slop of the trend and is assumed to be an AR(1) variation around a long-term slope of $\Delta_{\delta}$. $\gamma_t$ has the following specification:
\begin{align}
    \sum_{j=0}^6\gamma_{t-j} &= \eta_{\gamma,t}
\end{align}
where $\eta_{\gamma,t}\sim\mathcal{N}(0,\sigma_{\gamma}^2)$. We further assume that $\eta_{\mu,t},\eta_{\delta,t},\eta_{\gamma,t}$ are mutually independent. 

To fit the LDS model Kalman Filter, we rewrite it as:
\begin{align*}
    \xi_t = \begin{pmatrix}
        \mu_t \\
        \delta_t \\
        \gamma_t \\
        \gamma_{t-1} \\
        \gamma_{t-2} \\
        \gamma_{t-3} \\
        \gamma_{t-4} \\
        \gamma_{t-5} \\
        \gamma_{t-6}
    \end{pmatrix}, 
    &\ 
    F_t = F = \begin{pmatrix}
        1 & 1  & 0 & 0 & 0 & 0 & 0 & 0 & 0 \\
        0 & \rho_{\delta}  & 0 & 0 & 0 & 0 & 0 & 0 & 0 \\
        0 & 0 & -1 & -1 & -1 & -1 & -1 & -1 & 0 \\
        0 & 0 & 1 & 0 & 0 & 0 & 0 & 0 & 0 \\
        0 & 0 & 0 & 1 & 0 & 0 & 0 & 0 & 0 \\
        0 & 0 & 0 & 0 & 1 & 0 & 0 & 0 & 0 \\
        0 & 0 & 0 & 0 & 0 & 1 & 0 & 0 & 0 \\
        0 & 0 & 0 & 0 & 0 & 0 & 1 & 0 & 0 \\
        0 & 0 & 0 & 0 & 0 & 0 & 0 & 1 & 0
    \end{pmatrix} \\
    v_t = \begin{pmatrix}
        \eta_{\mu,t} \\
        \eta_{\delta,t} \\
        \eta_{\gamma,t} \\
        0 \\
        0 \\
        0 \\
        0 \\
        0 \\
        0 \\
    \end{pmatrix},
    &\ 
    B_t = B = \begin{pmatrix}
        0 & 0 \\
        (1-\rho_{\delta})\Delta_{\delta} & 0 \\
        0 & 0 \\
        0 & 0 \\
        0 & 0 \\
        0 & 0 \\
        0 & 0 \\
        0 & 0 \\
        0 & 0 
    \end{pmatrix},
    \ 
    x_t = \begin{pmatrix}
        1 \\
        \mathbbm{1}_{holiday}
    \end{pmatrix} \\
    y_t = \begin{pmatrix}
        g_t \\
        e_t 
    \end{pmatrix},
    &\ 
    D_t = D = \begin{pmatrix}
        0 & h_1 \\
        a & h_2
    \end{pmatrix},
    \ 
    w_t = \begin{pmatrix}
        w_{1,t} \\
        w_{2,t}
    \end{pmatrix} 
    , \ w_t\sim\mathcal{N}(0,R) \\
    & H_t = H = \begin{pmatrix}
        1 & 1 & 0 & 0 & 0 & 0 & 0 & 0 & 0 \\
        \pi & \pi & 0 & 0 & 0 & 0 & 0 & 0 & 0
    \end{pmatrix}
\end{align*}
The system paramter is $\theta\equiv(\hat{\xi}_{1,0}, P_{1,0}, \rho_{\delta}, \Delta_{\delta}, Q, R, a, h_1, h_2, \pi)$, where $Q$ takes the form:
\begin{align*}
    Q \equiv \begin{pmatrix}
        \sigma_{\mu}^2 & 0 & 0 & \cdots & \cdots & 0 \\
        0 & \sigma_{\delta}^2 & 0 & & & \vdots \\
        0 & 0 & \sigma_{\gamma}^2 & & & \vdots \\
        \vdots & & & 0 & & \vdots \\
        \vdots & & & & \ddots & \vdots \\
        0 & \cdots & \cdots & \cdots & \cdots & 0
    \end{pmatrix}
\end{align*}
We may use the standard EM algorithm to solve $\theta$.

\printbibliography
\pagebreak
\appendix

\section{Kalman Filter}
\subsection{Derivation of $\hat{\xi}_{t,t}$ and $P_{t,t}$} \label{ap:iter_proj}
\begin{lemma}[Law of Iterated Projections] \label{lem:1}
    Let $\mathcal{P}(Y_3|Y_2,Y_1)$ be the projection of $Y_3$ on $(Y_2, Y_1)$. Denote $\Omega_{ij}$ as $\Omega_{ij} = E(Y_iY_j^{'})$, then we have projections:
    \[
        \mathcal{P}(Y_3|Y_2,Y_1) = \mathcal{P}(Y_3|Y_1)+H_{32}H_{22}^{-1}[Y_2 - \mathcal{P}(Y_2|Y_1)]
    \]
    and variance matrix:
    \[
        Var(Y_3|Y_2,Y_1) = H_{33} - H_{32}H_{22}^{-1}H_{32}^{'}
    \]
    where 
    \begin{align*}
        H_{22} &= E\{[Y_2-\mathcal{P}(Y_2|Y_1)][Y_2-\mathcal{P}(Y_2|Y_1)]^{'}\} \\
        H_{23} &= H_{32}^{'} = E\{[Y_2-\mathcal{P}(Y_2|Y_1)][Y_3-\mathcal{P}(Y_3|Y_1)]^{'}\} \\
        H_{33} &= E\{[Y_3-\mathcal{P}(Y_3|Y_1)][Y_3-\mathcal{P}(Y_3|Y_1)]^{'}\}
    \end{align*}
\end{lemma}

Let $\xi_t$ as $Y_3$, $y_t$ as $Y_2$, and $(x_t,Y_{t-1})$ as $Y_1$, we obtain formula for $\hat{\xi}_{t,t}$ and $P_{t,t}$.

\subsection{Derivation of Joseph Form of $P_{t,t}$} \label{ap:joseph}
\begin{align*}
    P_{t,t} &= [I - K_tH_t]P_{t,t-1}[I - K_tH_t]' + K_tR_tK_t^{'} \\
    &= P_{t,t-1} - K_tH_tP_{t,t-1} - P_{t,t-1}H_t^{'}K_t^{'} + K_tH_tP_{t,t-1}H_t^{'}K_t^{'} + K_tR_tK_t^{'} \\
    &= P_{t,t-1} - K_tH_tP_{t,t-1} - P_{t,t-1}H_t^{'}K_t^{'} + K_t(H_tP_{t,t-1}H_t^{'} + R_t)K_t^{'} \\
    &= P_{t,t-1} - K_tH_tP_{t,t-1} - P_{t,t-1}H_t^{'}K_t^{'} + P_{t,t-1}H_t^{'}K_t^{'} \\
    &= P_{t,t-1} - K_tH_tP_{t,t-1}
\end{align*}
The fourth equality follows from equation (\ref{eq:gain}).

\subsection{Proof of Diffuse Kalman Filter} \label{ap:init_filter}
This section derives from Section 5 of (\cite{durbin_koopman_2001}), but provides an alternative formulation using the Joseph form. For the original proof without the Joseph formulation, please refer to (\cite{durbin_koopman_2001}). The key intuition is to give an exact result for the initial state distribution with arbitrarily large but finite variances, then find out the limit as the variances approach infinity. Given definition of $P_{t,t-1}$ in equation (\ref{eq:P_diffuse}), and linearity of Kalman filtering, we have:
\begin{align*}
    \Upsilon_t &= \kappa \Upsilon_{\infty,t} + \Upsilon_{*,t} + \mathcal{O}(\kappa^{-1}) \\
    \Upsilon_{\infty,t} &= H_tP_{\infty,t}H_t^{'} \\ 
    \Upsilon_{*,t} &= H_tP_{*,t}H_t^{'} + R_t
\end{align*}
We are interested in finding $\Upsilon_t^{-1}$. Expanding $\Upsilon_t^{-1}$ as a power series in $\kappa^{-1}$, we have:
\begin{align*}
    \Upsilon_t^{-1} = \Upsilon_t^{(0)} + \kappa^{-1}\Upsilon_t^{(1)} + \kappa^{-2}\Upsilon_t^{(2)}+\mathcal{O}(\kappa^{-3})
\end{align*}
where $\Upsilon_t^{(i)}$ is the $i$-th term associated with the power series expansion. We collapse other terms of the expansion to $\mathcal{O}(\kappa^{-3})$ as they will converge to 0 as $\kappa \rightarrow \infty$. We find $\Upsilon_t^{(i)}$ by using the fact that: 
\begin{align}
    I =& \Upsilon_t\Upsilon_t^{-1} \nonumber \\
    =& (\kappa \Upsilon_{\infty,t} + \Upsilon_{*,t} + \kappa^{-1}\Upsilon_{a,t} + \kappa^{-2}\Upsilon_{b,t}) \label{eq:inverse_diff}
 \\
    &\times (\Upsilon_t^{(0)} + \kappa^{-1}\Upsilon_t^{(1)} + \kappa^{-2}\Upsilon_t^{(2)}+\mathcal{O}(\kappa^{-3})) \nonumber
\end{align}

Solving equation (\ref{eq:inverse_diff}), we have:
\begin{align}
    \Upsilon_{\infty,t}\Upsilon_t^{(0)} = 0 \label{eq:upsilon_begin} \\
    \Upsilon_{*,t}\Upsilon_t^{(0)} + \Upsilon_{\infty,t}\Upsilon_t^{(1)} = I \\
    \Upsilon_{a,t}\Upsilon_t^{(0)} + \Upsilon_{*,t}\Upsilon_t^{(1)} + \Upsilon_{\infty,t}\Upsilon_t^{(2)} = 0 \label{eq:upsilon_end}
\end{align}
For a full treatment when $\Upsilon_t$ is not 1-by-1 matrix, see (\cite{koopman_1997}). If we are combine initialization with sequential filtering/smoothing, the solution to equation (\ref{eq:upsilon_begin}) to (\ref{eq:upsilon_end}) is much simpler\footnote{In the univariate case, non-singularity is equivalent to being non-zero, and singularity is equivalent to being zero.} and is shown below.

If $\Upsilon_{\infty,t}\neq 0$, we have:
\begin{align*}
    \Upsilon_t^{(0)} &= 0 \\
    \Upsilon_t^{(1)} &= \Upsilon_{\infty,t}^{-1} \\
    \Upsilon_t^{(2)} &= -\Upsilon_{\infty,t}^{-1}\Upsilon_{*,t}\Upsilon_{\infty,t}^{-1}
\end{align*}
By the fact that $K_t = P_{t,t-1}H_t^{'}\Upsilon_t^{-1}$, we can write $K_t$ as:
\begin{align}
    K_t &= K_t^{(0)} + \kappa^{-1}K_t^{(1)} + \mathcal{O}(\kappa^{-2}) \label{eq:K1_diffuse_start} \\
    K_t^{(0)} &= P_{\infty,t}H_t^{'}\Upsilon_t^{(1)} \\
    K_t^{(1)} &= P_{*,t}H_t^{'}\Upsilon_t^{(1)} + P_{\infty,t}H_t^{'}\Upsilon_t^{(2)} \label{eq:K1_diffuse_end}
\end{align}

Similarly, we can rewrite $\hat{\xi}_{t+1,t}$ as:
\begin{align}
    \hat{\xi}_{t,t} =& \hat{\xi}_{t,t-1} + K_t^{(0)}d_t + \mathcal{O}(\kappa^{-1}) \label{eq:diff_xi1} 
\end{align}
For $P_{t,t}$, we have:
\begin{align*}
    P_{t,t} =& [I-K_tH_t]P_{t,t-1}[I-K_tH_t]^{'}+K_tR_tK_t^{'} \\
    =& [I-(K_t^{(0)}+\kappa^{-1}K_t^{(1)})H_t](\kappa P_{\infty,t}+P_{*,t})[I-(K_t^{(0)}+\kappa^{-1}K_t^{(1)})H_t]^{'} \\ 
    &+ K_t^{(0)}R_tK_t^{(0)} + \mathcal{O}(\kappa^{-1}) \\
    =& \kappa(I-K_t^{(0)}H_t)P_{\infty,t}(I-K_t^{(0)}H_t)^{'} + (I-K_t^{(0)}H_t)P_{*,t}(I-K_t^{(0)}H_t)^{'} \\
    &-K_t^{(1)}H_tP_{\infty,t}(I-K_t^{(0)}H_t)^{'} - (I-K_t^{(0)}H_t)P_{\infty,t}H_t^{'}K_t^{(1)'} \\
    &+ K_t^{(0)}R_tK_t^{(0)} + \mathcal{O}(\kappa^{-1}) \\
    =& \kappa(I-K_t^{(0)}H_t)P_{\infty,t}(I-K_t^{(0)}H_t)^{'} + (I-K_t^{(0)}H_t)P_{*,t}(I-K_t^{(0)}H_t)^{'} \\
    &-K_t^{(1)}H_tP_{\infty,t} + K_t^{(1)}H_tP_{\infty,t}H_t^{'}K_t^{(0)'} \\
    &- P_{\infty,t}H_t^{'}K_t^{(1)'} + K_t^{(0)}H_tP_{\infty,t}H_t^{'}K_t^{(1)'}  \\
    &+ K_t^{(0)}R_tK_t^{(0)} + \mathcal{O}(\kappa^{-1}) 
\end{align*}
Note that $K_t^{(0)} = P_{\infty,t}H_t^{'}\Upsilon_t^{(1)} = P_{\infty,t}H_t^{'}(H_tP_{\infty,t}H_t^{'})^{-1}$, we have:
\begin{align*}
    K_t^{(0)}H_tP_{\infty,t}H_t^{'} &= P_{\infty,t}H_t^{'}(H_tP_{\infty,t}H_t^{'})H_tP_{\infty,t}H_t^{'} \\
    &= P_{\infty,t}H_t^{'}
\end{align*}
Now we have a clean Joseph form for the diffuse filter as:
\begin{align}
    P_{t,t}=& \kappa (I-K_t^{(0)}H_t)P_{\infty,t}(I-K_t^{(0)}H_t)^{'} \label{eq:diff_P1} \\
    &+(I-K_t^{(0)}H_t)P_{*,t}(I-K_t^{(0)}H_t)^{'} \nonumber \\
    &+K_t^{(0)}R_tK_t^{(0)'} + \mathcal{O}(\kappa^{-1}) \nonumber
\end{align}

If $\Upsilon_{\infty,t}=0$, then $H_tP_{\infty,t}=0$ by PSD properties. It is easier to get $\hat{\xi}_{t,t}$ and $P_{t,t}$:
\begin{align}
    K_t =& P_{*,t}H_t^{'}\Upsilon_{*,t}^{-1} + \mathcal{O}(\kappa^{-1}) \nonumber \\
    =& K_t^{(*)} + \mathcal{O}(\kappa^{-1}) \label{eq:K2_diffuse} \\
    \hat{\xi}_{t,t} =& \hat{\xi}_{t,t-1} + K_t^{(*)}d_t + \mathcal{O}(\kappa^{-1}) \label{eq:diff_xi2} 
\end{align}

For $P_{t,t}$, we have:
\begin{align}
    P_{t,t} =& \kappa (I-K_tH_t)P_{\infty,t}(I-K_tH_t)^{'} \nonumber \\
    &+ (I-K_t^{(*)}H_t)P_{*,t}(I-K_t^{(*)}H_t)^{'} \nonumber \\
    &+ K_t^{(*)}R_tK_t^{(*)'} + \mathcal{O}(\kappa^{-1}) \nonumber \\
    =& \kappa P_{\infty, t} + (I-K_t^{(*)}H_t)P_{*,t}(I-K_t^{(*)}H_t)^{'} \label{eq:diff_P2} \\
    &+ K_t^{(*)}R_tK_t^{(*)'} + \mathcal{O}(\kappa^{-1}) \nonumber
\end{align}

\subsection{Proof of the Degeneration Algorithm} \label{ap:transition}
Define $P_{\infty,t,t}$ as the diffuse part of $P_{t,t}$. From Appendix \ref{ap:init_filter}, we have:
\begin{align*}
    P_{\infty,t,t} = P_{\infty,t}[I - H_t^{'}(H_tP_{\infty,t}H_t^{'})^{-1}H_tP_{\infty,t}]
\end{align*}

Without loss of generality, we can re-arrange indices of $\xi_t$ such that:
\begin{align*}
    P_{\infty,1}=\begin{pmatrix}
        I_q & 0 \\
        0 & 0
    \end{pmatrix}
\end{align*}

In the univariate case, $H_t$ is a row vector, and $H_t^{'}H_tP_{\infty,t}$ is a similarly partitioned matrix as $P_{\infty, 1}$. Therefore, any change to the matrix only affects the non-zero partition, and we can focus solely on this partition. In what follows I will provide a proof in the case where $P_{\infty,1}$ has full rank (i.e. the initial condition is completely uninformative), and the conclusion is readily adapted to the mixed diffuse case where $P_{\infty,1}$ does not have full rank through partitioning operations and LDL decomposition of $P_{\infty,t}$. 

If $P_{\infty,t}=0$, $P_{\infty,t,t} = P_{\infty,t}$ and is not updated. If $P_{\infty,t}\neq0$, $H_t^{'}(H_tP_{\infty,t}H_t^{'})^{-1}H_tP_{\infty,t}$ is an idempotent matrix. If a matrix $A$ is idempotent, then:
\begin{boenumerate}
    \item $Trace(A)=rank(A)$
    \item $I-A$ is also idempotent
\end{boenumerate}

Using the fact that $Trace(AB)=Trace(BA)$, we have:
\begin{align*}
    Trace[H_t^{'}(H_tP_{\infty,t}H_t^{'})^{-1}H_tP_{\infty,t}] = Trace[H_tP_{\infty,t}H_t^{'}(H_tP_{\infty,t}H_t^{'})^{-1}] = 1
\end{align*}

Using the fact that $Trace(A+B) = Trace(A) + Trace(B)$, we have:
\begin{align*}
    Trace[I - H_t^{'}(H_tP_{\infty,t}H_t^{'})^{-1}H_tP_{\infty,t}]=q-1
\end{align*}

If $A$ has full rank, then by the fact that $Trace(AB)=Trace(B)$, we have:
\begin{align*}
    Trace(P_{\infty,t,t}) = q-1
\end{align*}

Therefore, if $\Upsilon_{\infty,t}\neq0$, the rank of $P_{\infty,t}$ is reduced by $1$ with each update. If after period $t$, $rank(P_{\infty,t,t})>0$, we have $P_{\infty,t+1} = F_tP_{\infty,t,t}F_t^{'}$. By the fact that $rank(AB)\leq min[rank(A), rank(B)]$, we have $q_{t+1}=min[rank(P_{\infty,t,t}),rank(F_t), rank(P_{\infty,t+1})]$. I use this updating rule instead of $q_{t} = rank(P_{\infty,t+1})$ to avoid numerical instability. 

\section{Kalman Smoother}
\subsection{Proof of State Smoothing} \label{ap:smooth}
The backbone of the proof is Lemma \ref{lem:1}. Note that knowing $\{Y_T, X_T\}$ is equivalent to knowing $\{Y_{t-1},d_t,d_{t+1},...,d_T, X_T\}$. In addition, $d_i \perp Y_{t-1}$ from the fact that $\hat{\xi}_{t,t-1}$ is a linear projection of $Y_{t-1}$ with Gaussian distribution. Applying this result iteratively, we can also obtain $d_i \perp d_j \:\forall\: \{i,j\} \in \{t,...,T\}$.

Now consider $\hat{\xi}_{t,T}$, denote $\Delta_t\equiv\{d_t,d_{t+1},...,d_T\}^{'}$ 
\begin{align*}
    \hat{\xi}_{t,T} &= E(\xi_t|Y_{t-1},\Delta_{t},X_T;\theta) \\
    &= \hat{\xi}_{t,t-1} + Cov(\xi_t,\Delta_t|Y_{t-1},X_T;\theta)Var(\Delta_t|Y_{t-1},X_T;\theta)^{-1}\Delta_t
\end{align*}

It is straightforward to show $Var(d_j|Y_{t-1},X_T;\theta)=\Upsilon_j$. For $Cov(\xi_t,d_j|Y_{t-1},X_T;\theta)$, we have:
\begin{align*}
    Cov(\xi_t,d_j|Y_{t-1},X_T;\theta) &= E(\xi_td_j^{'}|Y_{t-1},X_T;\theta) \\
    &= E[\xi_t(\xi_j-\hat{\xi}_{j,j-1})^{'}|Y_{t-1},X_T;\theta]H_j^{'} 
\end{align*}

If $j=t$, then:
\[
    E[\xi_t(\xi_t-\hat{\xi}_{t,t-1})^{'}|Y_{t-1},X_T;\theta] = P_{t,t-1}
\]

If $j=t+1$, then:
\begin{align*}
    E[\xi_t(\xi_{t+1}-\hat{\xi}_{t+1,t})|Y_{t-1},X_T;\theta] &= E[\xi_t(\xi_t-\hat{\xi}_{t,t-1}-K_td_t)^{'}F_t^{'}|Y_{t-1},X_T;\theta] \\
    &= P_{t,t-1}L_t^{'}
\end{align*}

For $j>t+1$, we have:
\begin{align*}
    E[\xi_t(\xi_{j}-\hat{\xi}_{j,j-1})|Y_{t-1},X_T;\theta] = P_{t,t-1}\prod_{i=t}^{j-1}L_i^{'}
\end{align*}

To compute $\hat{\xi}_{t,T}$, note that $Var(\Delta_{t}|Y_{t-1},X_T;\theta)$ is a block diagonal matrix, then we may express $\hat{\xi}_{t,T}$ as:
\begin{align*}
    \hat{\xi}_{t,T} &= \hat{\xi}_{t,t-1} + \sum_{j=t}^{T}Cov(\xi_t,d_j)\Upsilon_j^{-1}d_j \\
    &= \hat{\xi}_{t,t-1} + P_{t,t-1}\sum_{j=t}^T\left[\left(\prod_{i=t}^{j-1}L_{i}^{'}\right)H_j^{'}\Upsilon_j^{-1}d_j\right]
\end{align*}
If $j=t$, then we replace the product operation is not carried out and is replace with $I$. Define $r_t$ as:
\begin{align*}
    r_{t-1} \equiv \begin{cases}
        0, & t=T+1 \\
        \sum_{j=t}^T\left(\prod_{i=t}^{j-1}L_{i}^{'}\right)H_j^{'}\Upsilon_j^{-1}d_j, & t\leq T
    \end{cases}
\end{align*}

We can then write backwards recursive formulation for $\hat{\xi}_{t,T}$ as:
\begin{align*}
    r_{t-1} &= H_t^{'}\Upsilon_t^{-1}d_t + L_t^{'}r_t \\
    \hat{\xi}_{t,T} &= \hat{\xi}_{t,t-1} + P_{t,t-1}r_{t-1} 
\end{align*}

To calculate $P_{t,T}$, we use Lemma \ref{lem:1} again and have the following result:
\begin{align*}
    P_{t,T} &= P_{t,t-1} - \sum_{j=t}^TCov(\xi_t,d_j)\Upsilon_j^{-1}Cov(\xi_t,d_j)^{'} \\
    &=P_{t,t-1} - P_{t,t-1}\sum_{j=t}^{T}\left[\left(\prod_{i=t}^{j-1}L_{i}^{'}\right)H_j^{'}\Upsilon_j^{-1}H_j\left(\prod_{i=t}^{j-1}L_{i}^{'}\right)^{'}\right]P_{t,t-1}
\end{align*}

Similarly, we can find $P_{t,T}$ through backwards recursions. Define $N_t$ as:
\begin{align*}
    N_{t-1} \equiv \begin{cases}
        0, & t=T+1 \\
        \sum_{j=t}^{T}\left[\left(\prod_{i=t}^{j-1}L_{i}^{'}\right)H_j^{'}\Upsilon_j^{-1}H_j\left(\prod_{i=t}^{j-1}L_{i}^{'}\right)^{'}\right], & t\leq T
    \end{cases}
\end{align*}
The recursive formulation for $P_{t,T}$ is:
\begin{align*}
    N_{t-1} &= H_t^{'}\Upsilon_t^{-1}H_t + L_t^{'}N_tL_t \\
    P_{t,T} &= P_{t,t-1}- P_{t,t-1}N_{t-1}P_{t,t-1} 
\end{align*}

\section{Proof of Disturbance Smoothing} \label{ap:disturb_smooth}
The derivation of disturbance smoothers is similar to that of state smoothers in Appendix \ref{ap:smooth}. We have:
\begin{align*}
    \hat{v}_{t,T} &= \sum_{j=t}^{T}E(v_td_j^{'})\Upsilon_j^{-1}d_j \\
    \hat{w}_{t,T} &= \sum_{j=t}^{T}E(w_td_j^{'})\Upsilon_j^{-1}d_j \\
    V_t &= Q_t - \sum_{j=t}^{T}Cov(v_t,d_j^{'})\Upsilon_j^{-1}Cov(v_t,d_j^{'})^{'} \\
    W_t &= R_t - \sum_{j=t}^{T}Cov(w_t,d_j^{'})\Upsilon_j^{-1}Cov(w_t,d_j^{'})^{'} 
\end{align*}

First consider measurement errors $w_t$. Note that:
\begin{align*}
    E(w_td_j^{'}|Y_{t-1},X_T;\theta)=&Cov(w_t, d_j|Y_{t-1},X_T;\theta) \\
    =&E[w_t(\xi_j-\hat{\xi}_{j,j-1})^{'}|Y_{t-1},X_T;\theta]H_j^{'} \\ 
    &+ E(w_tw_j^{'}|Y_{t-1}X_T;\theta)
\end{align*}
We have:
\begin{align*}
    E(w_td_j^{'}|Y_{t-1},X_T;\theta) = \begin{cases}
        R_t, & j=t \\
        -R_tM_t^{'}H_{t+1}^{'}, & j=t+1 \\
        -R_tM_t^{'}\prod_{i=t+1}^{j-1}(L_i^{'})H_{j}^{'}, & j=t+2,...,T
    \end{cases}
\end{align*}

We can write $\hat{w}_{t,T}$ and $W_t$ as:
\begin{align*}
    \hat{w}_{t,T} &= R_t\Upsilon_t^{-1}d_t - R_tM_t^{'}\sum_{j=t+1}^{T}[\prod_{i=t+1}^{j-1}(L_i^{'})H_j^{'}\Upsilon_j^{-1}d_j] \\
    &= R_t(\Upsilon_t^{-1}d_t-M_t^{'}r_t) \\
    W_t &= R_t - R_t\{\Upsilon_t^{-1} + M_t^{'}\sum_{j=t+1}^{T}[\prod_{i=t+1}^{j-1}(L_i^{'})H_j^{'}
    \Upsilon_j^{-1}H_j\prod_{i=t+1}^{j-1}(L_i^{'})^{'}]M_t\}R_t \\
    &= R_t - R_t(\Upsilon_t^{-1}+M_t^{'}N_tM_t)R_t
\end{align*}

Next consider state transition errors $v_t$. Note that:
\begin{align*}
    E(v_td_j^{'}|Y_{t-1},X_T;\theta) &= Cov(v_t,d_j|Y_{t-1},X_T;\theta) \\
    &= E[v_t(\xi_j-\hat{\xi}_{j,j-1})^{'}|Y_{t-1},X_T;\theta]H_j^{'}
\end{align*}

We have:
\begin{align*}
    E(v_td_j^{'}|Y_{t-1},X_T;\theta) = \begin{cases}
        0, & j=t \\
        Q_tH_{t+1}^{'}, & j=t+1 \\
        Q_t\prod_{i=t+1}^{j-1}(L_i^{'})H_{j}^{'}, & j=t+2,...,T
    \end{cases}
\end{align*}

We can write $\hat{v}_{t,T}$ and $V_t$ as:
\begin{align*}
    \hat{v}_{t,T} &= Q_t\sum_{j=t+1}^{T}[\prod_{i=t+1}^{j-1}(L_i^{'})H_j^{'}\Upsilon_j^{-1}d_j] \\
    &= Q_tr_t \\
    V_t &= Q_t - Q_t\sum_{j=t+1}^{T}[\prod_{i=t+1}^{j-1}(L_i^{'})H_j^{'}\Upsilon_j^{-1}H_j\prod_{i=t+1}^{j-1}(L_i^{'})^{'}]Q_t \\
    &= Q_t - Q_tN_tQ_t
\end{align*}

Now we have the recursive formula for disturbance smoothers:
\begin{align*}
    \hat{w}_{t,T} &= R_t(\Upsilon_t^{-1}d_t - M_t^{'}r_t) \\
    \hat{v}_{t,T} &= Q_tr_t \\
    W_t &= R_t - R_t(\Upsilon_t^{-1}+M_t^{'}N_tM_t)R_t \\
    V_t &= Q_t - Q_tN_tQ_t 
\end{align*}
where $r_t$ and $N_t$ are defined in Appendix \ref{ap:smooth}.

\subsection{Proof of Diffuse Disturbance Smoothing} \label{ap:init_smoother}
For disturbance smoothers, we are interested in $v_t$ and $w_t$, which fortunately require less computational expenses than do smoothed estimates for $\xi_t$. In what follows I simplify the proof in (\cite{durbin_koopman_2003}) to better adapt to the need of disturbance smoothers in the univariate case\footnote{The difference lies in the magnitude of exponential expansion. Disturbance smoothers require less expansion than do state smoothers.If readers are interested in computing smoothed state variance, refer to (\cite{durbin_koopman_2003}) for details.}. 

Recall that $r_{t-1} = H_t^{'}\Upsilon_t^{-1}d_t + L_t^{'}r_t$. If $\Upsilon_{\infty,t}\neq0$, we can rewrite $r_{t-1}$ as:
\begin{align*}
    r_{t-1} =& r_{t-1}^{(0)} + \mathcal{O}(\kappa^{-1}) \\
    =& H_t^{'}(\kappa^{-1}\Upsilon_t^{(1)}+\kappa^{-2}\Upsilon_{t}^{(2)})(d_t^{(0)} + \kappa^{-1}d_t^{(1)}) \\
    &+ (L_t^{(0)} + \kappa^{-1}L_t^{(1)})(r_t^{(0)}+\kappa^{-1}r_t^{(1)}) + \mathcal{O}(\kappa^{-2})
\end{align*}
Rearrange the terms and we have the recursive formula for $r_{t-1}^{(0)}$:
\begin{align}
    r_{t-1}^{(0)} &= L_t^{(0)'}r_t^{(0)} \label{eq:diff_disturb1_start}
\end{align}
where $L_t^{(0)}=F_t(I-P_{\infty,t}H_t^{'}\Upsilon_{\infty,t}^{-1}H_t)$.

Similarly, we can rewrite $N_{t-1}$ as:
\begin{align}
    N_{t-1} &= N_{t-1}^{(0)} + \mathcal{O}(\kappa^{-1}) \nonumber \\
    N_{t-1}^{(0)} &= L_t^{(0)'}N_t^{(0)}L_t^{(0)} 
\end{align}

Now we can calculate the diffuse disturbance smoother for $\Upsilon_{\infty,t}\neq0$ as:
\begin{align}
    \hat{w}_{t:T} &= -R_tK_t^{(0)'}F_t^{'}r_t^{(0)} \\
    \hat{v}_{t:T} &= Q_tr_t^{(0)} \\
    W_t &= R_t - R_tK_t^{(0)'}F_t^{'}N_t^{(0)}F_tK_t^{(0)}R_t \\
    V_t &= Q_t - Q_tN_t^{(0)}Q_t \label{eq:diff_disturb1_end}
\end{align}


If $\Upsilon_{\infty,t}=0$, we can rewrite $r_{t-1}$ as:
\begin{align*}
    r_{t-1} =& r_{t-1}^{(0)} + \mathcal{O}(\kappa^{-1}) \\
    =& H_t^{'}\Upsilon_{*,t}^{-1}d_t + L_{*,t}^{'}r_{t}^{(0)} + \mathcal{O}(\kappa^{-1})
\end{align*}
Rearrange the terms and we have the recursive formula for $r_{t-1}^{(0)}$:
\begin{align}
    r_{t-1}^{(0)} &= H_t^{'}\Upsilon_{*,t}^{-1}d_t + L_{*,t}^{'}r_{t}^{(0)} \label{eq:diff_disturb0_start}
\end{align}
where $L_{*,t}=F_t(I-P_{*,t}H_t^{'}\Upsilon_{*,t}^{-1}H_t)$.

Similarly, we can rewrite $N_{t-1}$ as:
\begin{align}
    N_{t-1} &= N_{t-1}^{(0)} + \mathcal{O}(\kappa^{-1}) \nonumber \\
    N_{t-1}^{(0)} &= H_t^{'}\Upsilon_{*,t}^{-1}H_t + L_{*,t}^{'}N_t^{(0)}L_{*,t} 
\end{align}

Now we can calculate the diffuse disturbance smoother for $\Upsilon_{\infty,t}=0$ as:
\begin{align}
    \hat{w}_{t:T} &= R_t(\Upsilon_{*,t}^{-1}d_t - K_t^{(*)'}F_t^{'}r_t^{(0)}) \\
    \hat{v}_{t:T} &= Q_tr_t^{(0)} \\
    W_t &= R_t - R_t(\Upsilon_{*,t}^{-1} + K_t^{(*)'}F_t^{'}N_t^{(0)}F_tK_t^{(*)})R_t \\
    V_t &= Q_t - Q_tN_t^{(0)}Q_t \label{eq:diff_disturb0_end}
\end{align}









\section{Proof of Sequential Smoothing}

\section{EM Algorithm Premier:} \label{ap:EM_proof}
Denote $Y$ as observed measurements, $\xi$ as hidden states, and $\theta$ as parameters to be estimated. We want to optimize:
\begin{align}
    L(\theta) & = log[P(Y|\theta)] \nonumber \\
    & = log\left[\frac{P(Y,\xi|\theta)}{P(\xi|Y,\theta)}\right] \nonumber \\
    & = log[P(Y,\xi|\theta)] - log[P(\xi|Y,\theta)] \label{eq:general_mle}
\end{align}

Take expectation of equation (\ref{eq:general_mle}) wrt. some distribution of $\xi$ with pdf $f(\xi)$ and get:

\begin{align}
    L(\theta) = & \int f(\xi)log[P(Y,\xi|\theta)]d\xi \nonumber \\
    & - \int f(\xi)log[P(\xi|Y,\theta)]d\xi \nonumber
\end{align}
To optimize $L(\theta)$ we can iterate through the E steps and M steps to achieve a local maximum. By Jensen's inequality, we have the second term in equation (\ref{eq:general_mle}) maximized when $f(\xi)=P(\xi|Y,\theta)$ (E-step). If we define:
\begin{align}
    Q(\theta) = \int log[P(Y,\xi|\theta)]f(\xi|Y,\theta)d\xi \label{eq:Q}
\end{align}
then maximizing $Q(\theta)$ is equivalent to maximizing $ L(\theta)$. For a given $\hat{\theta}$ and $P(\xi|Y, \hat{\theta})$, we find $\theta$ to optimize the first term (M-step). Use the new $\theta$ as $\hat{\theta}$ for the next iteration, and we will reach a local maximum. It is important to note that for a given $\hat{\theta}$, $f(\xi|Y, \hat{\theta})$ is a given quantity and does not change wrt. $\theta$. 

\section{Derivation of Log-likelihood for $\Xi_T$} \label{ap:log}
If $t>1$, recall that $\delta_t \equiv \xi_t - F_{t-1}\xi_{t-1} - B_{t-1}x_{t-1}$. Expanding the expectation terms in equation (\ref{eq:log1_trace}), and denoting $\Theta \equiv (Y_T,X_T, \theta_i)$, we have:
\begin{align*}
    E(\delta_t\delta_t^{'}|\Theta) &= E[(\xi_t-F_{t-1}\xi_{t-1}-B_{t-1}x_{t-1}) 
    (\xi_t-F_{t-1}\xi_{t-1}-B_{t-1}x_{t-1})^{'}|\Theta] \\
    &= E(\xi_t\xi_t^{'}|\Theta) - F_{t-1}E(\xi_{t-1}\xi_{t}^{'}|\Theta) - B_{t-1}x_{t-1}E(\xi_t^{'}|\Theta) \\
    &- E(\xi_t\xi_{t-1}^{'}|\Theta)F_{t-1}^{'} + F_{t-1}E(\xi_{t-1}\xi_{t-1}^{'}|\Theta)F_{t-1}^{'}
    +B_{t-1}x_{t-1}E(\xi_{t-1}^{'}|\Theta)F_{t-1}^{'} \\
    &- E(\xi_t|\Theta)x_{t-1}^{'}B_{t-1}^{'} + F_{t-1}E(\xi_{t-1}|\Theta)x_{t-1}^{'}B_{t-1}^{'}
    +B_{t-1}x_{t-1}x_{t-1}^{'}B_{t-1}^{'}
\end{align*}
We already derive expression for $E(\xi_t|\Theta)$ in (\ref{eq:filter_begin}). In addition, we need to calculate $E(\xi_t\xi_t^{'}|\Theta)$ and $E(\xi_t\xi_{t-1}^{'}|\Theta)$. $E(\xi_t\xi_t^{'}|\Theta)$ is easy to  get:
\begin{align*}
    E(\xi_t\xi_t^{'}|\Theta) & = E(\xi_t|\Theta)E(\xi_t^{'}|\Theta) + Var(\xi_t|\Theta) \\
    &= \hat{\xi}_{t,T}(\hat{\xi}_{t,T})^{'} + P_{t,T}
\end{align*}
To find $E(\xi_t\xi_{t-1}^{'}|\Theta)$, consider the following\footnote{This derivation adapts the general strategy in deriving $\hat{\xi}_{t,T}$ in Chapter 13 of (\cite{hamilton_1994}).}:
\begin{align}
    E(\xi_t\xi_{t-1}^{'}|\xi_t,y_{t-1},X_T,\theta_i) &= \xi_tE(\xi_{t-1}^{'}|\xi_t,y_{t-1},X_T,\theta_i) \nonumber \\
    &= \xi_t(\hat{\xi}_{t-1,t-1})^{'} + \xi_t\xi_t^{'}J_{t-1}^{'} - \xi_{t}(\hat{\xi}_{t,t-1})^{'}J_{t-1}^{'} \label{eq:xi_t,t-1}
\end{align}
By the nature of Markov Dynamics, adding $\{y_t, y_{t+1}, ..., y_T\}$ does not change the value of $E(\xi_t\xi_{t-1}^{'}|\xi_t,y_{t-1},x_{t-1},\theta_i)$. Taking expectations of equation (\ref{eq:xi_t,t-1}) over $\xi_t$, we have:
\begin{align*}
    E(\xi_t\xi_{t-1}^{'})|\Theta) &= \hat{\xi}_{t,T}(\hat{\xi}_{t-1,t-1})^{'} + E(\xi_t\xi_t^{'}|\Theta)J_{t-1}^{'}
    -\hat{\xi}_{t,T}(\hat{\xi}_{t,t-1})^{'}J_{t-1}^{'} \\
    &= \hat{\xi}_{t,T}(\hat{\xi}_{t-1,t-1})^{'} + [\hat{\xi}_{t,T}(\hat{\xi}_{t,T})^{'} + P_{t,T} 
    -\hat{\xi}_{t,T}(\hat{\xi}_{t,t-1})^{'}]J_{t-1}^{'} \\
    &= \hat{\xi}_{t,T}(\hat{\xi}_{t-1,t-1})^{'} + [P_{t,T} + \hat{\xi}_{t,T}(\hat{\xi}_{t,T}-\hat{\xi}_{t,t-1})^{'}]J_{t-1}^{'}
\end{align*}
If $t=1$, we have:
\begin{align*}
    E(\delta_1\delta_1^{'}) &= E(\xi_1\xi_1^{'}|\Theta) - \hat{\xi}_{1,0}(\hat{\xi}_{1,T})^{'}
    -\hat{\xi}_{1,T}(\hat{\xi}_{1,0})^{'} + \hat{\xi}_{1,0}(\hat{\xi}_{1,0})^{'}
\end{align*}

\section{Derivation of Log-likelihood for $Y_T$} \label{ap:log2}
Recall that $\chi_t \equiv y_t - H_t\xi_t - D_tx_t$. Expanding the expectation terms in equation (\ref{eq:log2_trace}), and denoting $\Theta \equiv (Y_T,X_T,\theta_i)$, we have:
\begin{align*}
    E(\chi_t\chi_t^{'}|\Theta) &= E[(y_t - H_t\xi_t - D_tx_t)(y_t - H_t\xi_t - D_tx_t)^{'}|\Theta] \\
    &= (y_t-D_tx_t)(y_t-D_tx_t)^{'} - H_tE(\xi_t|\Theta)(y_t-D_tx_t)^{'} \\
    &-(y_t-D_tx_t)E(\xi_t^{'}|\Theta)H_t^{'} + H_tE(\xi_t\xi_t^{'}|\Theta)H_t^{'}
\end{align*}
Now look at equation (\ref{eq:log2_trace}) again. Let $M_t$ be:
\begin{align*}
    M_t \equiv Tr[R_t^{-1}E(\chi_t\chi_t^{'}|Y_T,X_T,\theta_{i-1})] 
\end{align*}
If $y_t$ has no missing measurements, it is straightforward to calculate $M_t$. If $y_t$ has missing measurements, but $R_t^{-1}$ is diagonal, we can follow (\cite{shumway_stoffer_1982}) and modify $R_t$, $H_t$, $D_t$ and $\chi_t$, which is done in section (\ref{sec:filter}). In fact, we are calculating the marginal distribution of the observed components. 
\end{document}
